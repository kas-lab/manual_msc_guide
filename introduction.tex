\chapter{Introduction}\label{c:introduction}

\section{Who is this guide for?}

This document has been written for master students intending to graduate at the KAS Laboratory, section Robot Dynamics of the department Cognitive Robotics (Faculty 3mE, TU Delft). Typically, these include MSc students from the MSc Robotics \url{https://www.tudelft.nl/onderwijs/opleidingen/masters/robotics/msc-robotics/programme} or BioMechanical Engineering (specialisations BioRobotics). Exceptionally, we supervise students from other backgrounds as well. 
Nonetheless, the guide contains a lot of general tips and tricks that can be applied in other domains.

Disclaimer: this guide is heavily based on the "Master Thesis Guide
Written by Delft Haptics Lab members" (version June 9 2020). We are very grateful to them for sharing such a beautiful guide, and hope there are ok with us reusing parts of it for the KAS Lab guide.

\section{Why this guide?}

A graduation project is a \textit{process}, and as a student typically you have not done it before.
This process covers most of your second year of your MSc, from finding a thesis supervisor and a topic, to eventuelly presenting and defending your project.

This guide is intended to help you in that process. 

% FROM HRI Manual
% Note that an MSc thesis project is not a straightforward affair to which a step-by-step ‘manual’ can be applied. As such, this guide is no more than a guide, it provides a structured approach, examples, and helpful tips \& tricks, but that does not mean this is the only way to go about your thesis project. Please also make use of other guides, tips, tricks, conversations with students & staff, and of course your own creativity and common sense.

\section{Overview of the graduation process and this guide}

\subsection{Graduation process - COPIED FROM HRI manual}
The second year of your MSc consists of three main parts. The three parts are:

\begin{itemize}
    \item Internship: The internship gives you a chance to explore academia or industry outside TU Delft, although you can also do a research assignment within the department. The topic does not need to align with the rest of your thesis project. You will be graded by a short report (10 pages), with a pass or fail.
    \item Literature study: During the literature study, you will need to define a relevant research question that can be answered with literature only, and answer it. This requires you to read and analyze scientific literature, refining your research question. You will be graded on a literature report, typically 30-50 pages.

    \item Master thesis: The master thesis is your final project at the TU Delft. During the project, you will define your own research question and use the proper scientific methods to answer it.

\end{itemize}

\subsection{Additional resources}
INCLUDE WHERE TO FIND OTHER resources, for example our TEAMS or DEPARTMET AND MSC ROBOTICS WEBSITES

\begin{itemize}
    \item The must-reads for every human-interaction researcher, grouped per topic. - TODO select in our Zotero.
    \item Examples of good literature reports.
    \item Examples of good MSc. reports.
    \item Examples of good MSc. presentations.
\end{itemize}

\subsection{Structure of this guide}

\begin{itemize}
    \item Chapter \ref{c:literature}
    \item Chapter \ref{c:workflow-tools}
    \item Chapter \ref{c:msc_thesis}
    \item Chapter \ref{}
\end{itemize}